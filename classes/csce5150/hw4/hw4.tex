\documentclass[12pt,twoside,letterpaper]{article}
\setlength\topmargin{0in}
\setlength\headheight{0in}
\setlength\headsep{0in}
\setlength\textheight{9.0in}
\setlength\textwidth{6.5in}
\setlength\oddsidemargin{0in}
\setlength\evensidemargin{0in}
\setlength{\parskip}{12pt}
\pagestyle{empty}
\raggedright
\begin{document}
Cameron Palmer - CSCE 5150\\
November 1, 2007
\section*{Homework 3}
\subsection*{Page 364 15-1}
Intuition - Scanning from left to right you connect each point to its closest neighbor. As the scan of the graph occurs you pick up a new point and connect it to the vertex closest to the point that is already in the bitonic tour.

If $Z$ represents the cost of the shortest bitonic path from i, through all verticies, and $1 \le i \le n$, $j > i$ then
$$Z_{i,j} = d(i,i+1) + Z_{i+1,j} \textrm{ where } j \ne i+1$$
which represents the addition of a new point to the tour, and
$$Z_{i,i+1} = \min_{k>i+1}{d(i,k) + Z_{i+1,k}}$$
It takes $\textrm{O}(i)$ to compute $Z_{i,j}$. So time complexity is $\Theta(\sum_{i=1}^n i)$ which is $\Theta(n^2)$.

\subsection*{Page 367 15-4}
For any $x \in T$. Let $W_x$ be the value of an optimal solution if we force $x$ in the the solution. Let $W_x^\prime$ be the value of the optimal solution for subtree rooted at $x$ if we force x out of the solution. Let $O_x$ be a value of an optimal solution for a subtree rooted at $x$. We want to compute $O$.

If $x$ is not a leaf,
\begin{equation}
O_x = \max(W_x,W_x^\prime)
\end{equation}
Let $C(x)$ be the children of $x$,
\begin{equation}
W_x = w(x) + \sum_{y \in C(x)} W_y^\prime
\end{equation}

\begin{equation}
W_x^\prime = \sum_{y \in C(x)} O_y
\end{equation}

If $x$ is a leaf,
$$W_x= W(x)$$
$$W_x^\prime = O$$

The algorithm is post order in which recurrence relations (1), (2), (3) are used to compute $O_x, W_x, W_x^\prime$ for all $x \in V$. Time complexity is $\textrm{O}(n)$ where $n = |v|$.

\subsection*{Page 369 15-7}
You could order the jobs $a$ by deadline $d$. From here you place the maximum profit $p$ in the matrix and optimize your path through the jobs. Which would yield $\textrm{O}(n^3)$ running time based upon the size of the table $n^2$ and then the recovery of the optimal path $n$.

In table $T$,
\[ T[i,t] = \max \left\{ \begin{array}{ll}
T[i-1,t]\\
T[i-1,t-t_i]+p_i & \mbox{if $t \le d_i$}\\
T[i-1,t-t_i] & \mbox{if $t > d_i$}\end{array} \right. \] 

\subsection*{Problem 4a}
Using Floyd-Warshall as a basis simply recursively follow the backpointers for all the shortest paths that had the same weight.

sub count(i, j)
	if ( edge(i, j) = d(i, j) ) // meaning if the edge and the shortest path are equivalent we are done.
		return 1
	else
		foreach edge(i, k): where d(i, k) + d(k, j) = d(i, j)
		sum += count(i, k)
		return sum

\subsection*{Problem 4b}
Since a fixed edge means that there is some edge e(i, j) we can easily lookup the j-th row and determine how many backpointers refer back to that cell.

\end{document}

