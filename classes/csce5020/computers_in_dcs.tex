\documentclass[12pt,twoside,letterpaper]{article}
\setlength\topmargin{0in}
\setlength\headheight{0in}
\setlength\headsep{0in}
\setlength\textheight{9.0in}
\setlength\textwidth{6.5in}
\setlength\oddsidemargin{0in}
\setlength\evensidemargin{0in}
\setlength{\parskip}{12pt}
\pagestyle{empty}
\raggedright
\begin{document}
Cameron Palmer - CSCE 5020\\
October 26, 2007
\section*{One Laptop Per Child}
The concept of computers in the classroom does not work. With all the slideshows, videos, interactive programs, and other visual eye candy, are we better off? I believe not. However, this does not mean that computers have no place in education or society. The real challenges facing Developing Countries could in part be addressed with the idea of One Laptop Per Child (OLPC). Nicholas Negroponte, co-founder of the MIT Media Laboratory and chairman of the non-profit OLPC, has the right combination of vision and skills to make a significant improvement to the those who most often cannot afford education. A personal computer can bring global contacts, perspectives, and information to an otherwise cut-off student.

For thousands of years, humans have distinguished themselves through their use of tools. Therefore, distributing the knowledge of our tools is a human imperative. Many of the world's Developing Countries, for example India, have regions or cities that already have access to computers and the Internet, but there are billions in rural areas who do not. Half of the world’s population lives on less than two dollars per day. This level of income makes the computers we use in the western world unattainable, even with free software. And even if computers could be delivered to every person in the world, the prerequisites of power and networks would make them useless. Therefore, developing tools for such a challenging environment should be of the utmost importance if we want to address education where the delivery of education is the hardest. 

Rather than to print and ship costly paper books that are susceptible to the elements, we could ship entire libraries of eBooks to the students. Any computerized book must contain screen technology that is far more advanced than the current off-the-shelf LCD panels. Anyone who has looked at a computer monitor for more than a few minutes knows that reading a book on current monitors is painful. Furthermore, current mainstream monitors are power-hungry, consuming half of the energy used by the computer, even in the most energy-efficient systems. Because of regular monitors’ high power consumption, we need electronic paper to make electronic text books possible. 

One electronic paper technology called eInk has been developed at MIT's Media Lab and is now a separate company (http://www.eink.com/). The revolutionary nature of this display technology is difficult to comprehend for someone who has not seen the display in person, but you can look for Sony's Reader PRS-505, which is an eBook that can display text and PDFs using eInk technology. An eInk screen looks like a document held in a picture frame that is 1.2 millimeters thick, and it only consumes power when you turn the page. eInk is the sort of technological innovation that can deliver basic knowledge tools to Developing Countries. OLPC's current prototypes use this type of technology in the form of a dual mode display. In one mode, the OLPC laptop functions like a regular color LCD consuming about 1.0 Watt per hour, and in a second mode it becomes a grayscale eInk-like display that consumes only 0.1 Watt per hour.

Humans are at their most creative and are most capable of learning when they are young. Yet during this period, if they get an education at all, they are usually subjected to an educational system that is a product of the industrial age\cite{1245651}—a factory of learning, in the worst sense. These teaching styles that are so effective in jump starting economies stifle creative thought. Only official state-sanctioned knowledge is taught in the critical primary education system and, as we know, there are many examples where states inject hate, religion, and prejudice into the classroom. If OLPC can be delivered, then we are providing the tool that with a little effort can easily supplement or augment the official school materials with external viewpoints and allow children to pursue their ideas and interests more fully and truthfully.

For the students to achieve the goal of personal exploration, a network is important. This network is needed in a place where even analog phones are scarce. In parts of India, phones are so scarce that the postal service gives the delivery person a cellphone to use as a mobile pay phone. Clearly, networks for the developing world will have to be addressed, and not in the sense of traditional expensive infrastructure\cite{303858}. Ad-hoc wireless networks built by the wireless technology in the laptop can deliver components of that network. Larger groups of people will be able to communicate, and with the addition of one wireless uplink, the entire village can be online. When even networking is not possible, information can be distributed in the mail via a single flash memory card that is nearly indestructible and difficult to censor.

Power consumption of the laptop is probably the most critical aspect of any computing initiative in a Developing Country. Taking great care when developing each component of OLPC, we can manage power consumption to the point where only a hand crank or pull string is required to power the machine. The goal of OLPC is 10:1 power return. Crank for one minute and the laptop should run for ten minutes. To put power consumption in perspective a typical laptop can take 10-40 Watts while the design target for OLPC is 2.0 Watts.

Any distribution method for a laptop computer that will be given away to children will face problems with a grey, or even a black, market. OLPC needs to avoid the situation that occurred in Brazil, where every child is entitled to free shoes: often the children go to school without shoes because the parents sell them on the black market. A distinctive design, child-sized components, and limited usefulness outside of education will help address these issues.

These computers must be waterproof, dust-proof, theft-deterrent, and drop-proof. The applications must work within extremely limited hardware, but deliver networking and other advance applications. Developers of OLPCs cannot expect to take advantage of Moore's Law to pack more performance into the laptop, but instead cut component costs. Efficient use of resources really underlines the whole project, or any similar project. It goes against our western nature and requires design skill sets that really do not exist in the mass-produced personal computer industry. This laptop must cost under \$100 to succeed. We need to look at countries like India and realize that while parts of the country are technologically advanced, 350-400 million people live below the poverty line, and 75 percent of them in rural areas. This puts the vast majority outside of the infrastructure that is a prerequisite of technological infrastructure. To tackle such wide spread poverty, we will need to deliver education and infrastructure, and a component of that infrastructure could be tools like OLPC.
\bibliographystyle{plain}
\bibliography{computers_in_dcs}
\end{document}
