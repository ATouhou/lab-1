\documentclass[12pt,twoside,letterpaper]{article}
\usepackage{amsmath}
\usepackage{amsthm}
\usepackage{amsfonts}
\usepackage{amssymb}
\setlength\topmargin{0in}
\setlength\headheight{0in}
\setlength\headsep{0in}
\setlength\textheight{9.0in}
\setlength\textwidth{6.5in}
\setlength\oddsidemargin{0in}
\setlength\evensidemargin{0in}
\setlength{\parskip}{12pt}
\pagestyle{empty}
\raggedright
\begin{document}
\section*{Vertex Cover}
$S \subseteq V$ is called a vertex cover if for any edge $ab \in E$ either $a \in S$ or $b \in S$. Minimum vertex cover is one with smallest number of vertices. This problem is NP-hard.

\subsection*{Approximation Algorithm}
If you choose nodes with maximum degree first. Be careful with a greedy algorithm because it might choose $a$ and $b$ first and then end up with degree 4 rather than minimal of degree 3.

Need to pick the minimum number of vertices that cover all the edges.

Let $G=(V,E)$ be a graph.
Let $M \subseteq E$. We say that $M$ is matching if no two edges in $M$ have an end point in common. For example a hexagon's maximum matching has cardinality of three.

Let $M \subseteq E$ be matching. We say $M$ is "maximal" if for any $e$ not in $M$ maximum matching $M \cup \{ e \}$ is not maximal matching.

Let $S$ be a vertex cover and $M$ be matching then $|S^*| \ge |M|$.

If this is maximal matching then you can pick up endpoints for vertex cover. Must pick up all $a$'s and $b$'s.

Let $M=\{a_1,b_1,a_2,b_2,...,a_k,b_k\}$ be a maximal matching then $S=\{a_1,..,a_k,b_1,...,b_k\}$ is a vertex cover. Note that $|S|=2|M|$\\
$2|S^*| \ge |S| = 2|M| \ge |S^*|$ which is an upper bound.
\end{document}