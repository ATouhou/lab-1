\documentclass[12pt,twoside,letterpaper]{article}
\setlength\topmargin{0in}
\setlength\headheight{0in}
\setlength\headsep{0in}
\setlength\textheight{9.0in}
\setlength\textwidth{6.5in}
\setlength\oddsidemargin{0in}
\setlength\evensidemargin{0in}
\setlength{\parskip}{12pt}
\pagestyle{empty}
\raggedright
\begin{document}
Cameron Palmer - CSCE 5020\\
October 26, 2007
\section*{One Laptop Per Child}
Computers in the classroom, don't work. For all the slideshows, videos, interactive programs, and other visual eye candy are we better off? I believe not. However, this doesn't mean computers have no place in education or society. The real challenges faced in Developing Countries could in part be addressed with the idea of One Laptop Per Child (OLPC). Nicholas Negroponte, co-founder of the MIT Media Laboratory, and chairman of the non-profit OLPC has the right combination of vision and skills to make a significant improvement to the those that most often cannot afford education. A personal computer can bring global contacts, perspectives, and information to its owner through the power of the Internet or the postal service and flash memory.

Since the earliest times we have distinguished ourselves through our tools, and distributing the knowledge of our tools is a human imperative. The world's developing countries like India have regions or cities that already have access to computers and the Internet, but their are billions in rural areas who don't. Half of the world lives on less than two dollars per day which makes the computers we use, even with free software unattainable. Even if computers could be delivered to every person in the world, the prerequisites of power, and networks would make them useless. So to develop tools for such a challenging environment should be of the utmost importance if we want to address education where the delivery of education is the hardest. 

Rather than print and ship costly paper books that are susceptible to the elements we could ship entire libraries of eBooks to the students. Any computerized book must contain screen technology that is far more advanced than the current off-the-shelf LCD panels. Anyone who has looked at a computer monitor for more than a minute knows reading a book on current monitors is painful. Further, current mainstream monitors are power hungry, representing half of some of the most energy efficient systems. We need electronic paper to provide make electronic text books possible. 

One electronic paper technology called eInk has been developed at MIT's Media Lab and is now a seperate company (http://www.eink.com/). The revolutionary nature of this display technology is difficult to comprehend for someone who hasn't seen the display in person, but you can look for Sony's Reader PRS-505 which is an eBook that can display text and PDFs using eInk technology. When you look at an eInk screen it looks like a document held in a picture frame that is 1.2 millimeters thick, and only consumes power when you turn the page. eInk is the sort of technological innovation that can deliver basic knowledge tools to Developing Countries. OLPC's current prototypes use this type of technology in the form of a dual mode display. In one mode the OLPC laptop functions like a regular color LCD consuming about 1.0 Watt, and in a second mode becomes a grayscale eInk-like display that consumes only .1 Watt.

Humans are at their most creative, and capable of learning when they are young. Yet during this period, if they get an education at all, are subjected to an educational system that is a product of the industrial age\cite{1245651}. A factory of learning, in the worst sense. These teaching styles that are so effective in jump starting economies stifle creative thought. Only official state sanctioned knowledge is taught in the critical primary education system and as we know there are many examples where states inject hate, religion, and prejudice into the classroom. If OLPC can be delivered, then we are providing the tool that with a little effort can easily supplement or augment the official materials with external viewpoints, and allow children to pursue their ideas and interests more fully and truthfully.

To achieve the goal of personal exploration a network is important. This network is needed in a place where even analog phones are scarce. In India phones are so scarce in parts of the country that the postal service gives the delivery person a cellphone to use as a mobile payphone. So networking in a laptop for the developing world will have to be addressed, and not in the traditional expensive infrastructure method\cite{303858}. Ad-hoc wireless networks built by the wireless technology in the laptop can deliver components of that network. Larger groups of people will be able to communicate and with the addition of one wireless uplink the entire village can be online. When even networking isn't possible information can be distributed in the mail via a single flash memory card that is nearly indestructable, and difficult to censor.

Power consumption of the laptop is probably the most critical aspect of any computing initiative in a Developing Country. If the village doesn't have power how will any of this work? With care in development of each component of OLPC we can manage power consumption to the point where only a hand crank, or pull string is required. The goal of OLPC is 10:1 power return. Crank for one minute and the laptop should run for ten minutes. To put power consumption in perspective a typical laptop can take 10-40 Watts while the design target for OLPC is 2.0 Watts.

Any design for a laptop computer that will be given away to children at cost will without fail find a grey or even a black market. OLPC needs to avoid the situation in Brazil where every child is entitled to free shoes, and often the children go to school without shoes because they are sold by the parents on the black market. A distinctive design, child-sized components, and limited usefullness outside of education will help address these issues.

These computers must be waterproof, dust-proof, theft-deterrent and drop-proof. The applications must work within extremely limited hardware, but deliver networking and other advance applications. Developers of OLPCs cannot expect to take advantage of Moore's Law to pack more performance into the laptop, but instead cut component costs. Efficient use of resources really underlines the whole project, or any similar project. It goes against our western nature, and requires design skill sets that really don't exist in the mass produced personal computer industry. This laptop must cost under \$100 to succeed. We need to look at countries like India and realize while parts are technologically advanced 350-400 million people are below the poverty line, and 75 percent of them in rural areas. This puts the vast majority outside of the infrastructure that is a prerequisite of technological infrastructure. To tackle such wide spread poverty we will need to deliver education and infrastructure, and a component of that infrastructure could be tools like OLPC.
\bibliographystyle{plain}
\bibliography{computers_in_dcs}
\end{document}
